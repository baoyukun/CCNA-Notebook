\documentclass[crop=false]{standalone}

\begin{document}
The focus of this course is on learning the fundamentals of networking.We will do the following:
\begin{itemize}
  \item Examine human versus network communication and see the parallels between them
  \item Be introduced to the two major models used to plan and implement networks: \textbf{OSI} and \textbf{TCP/IP}
  \item Gain an understanding of the "layered" approach to networks
  \item Become familiar with the various network devices and network addressing schemes
  \item Discover the types of media used to carry data across the network
  \item Be able to build simple LANs, perform basic configurations for routers and switches, and implement IP addressing schemes.
\end{itemize}

Our studying is based on the following environment:

\begin{itemize}
  \item System: \textbf{Ubuntu 16.04 LTS}\\
        installed on virtual machine "Oracle VM VirtualBox"\footnote{\url{https://www.virtualbox.org}}
  \item Tools:\begin{itemize}
        \item \textbf{Packet tracer 7.0} by Cisco Academy Community
        \item \textbf{Wireshark} Network traffic analyzer
        \item \textbf{XTerm} standard terminal emulator for the X window system
        \item \textbf{Netkit-ng/\acrshort{uml}} manipulation of virtual machines in \acrfull{uml}
        \footnote{\url{https://netkit-ng.github.io}}
        \end{itemize}
  \item Optional system image: \textbf{RES101-Ubuntu-2017}\footnote{User name: res101; Password: res101-2017}\\
        It integrates all necessary tools and is accessible on the Google Drive
        \footnote{\url{https://drive.google.com/open?id=0Byh-un6Ly9NCWERmcXhJN2RlYVU}}.
\end{itemize}

\addcontentsline{toc}{subsection}{Familiar with virtual machine commands}
\subsection*{Familiar with virtual machine commands}
Here are the useful commands for manipulating virtual machines.

\begin{tcolorbox}
\begin{flushleft}
  \textbf{\gls{lstart}}: start the whole network, machine by machine\\
  \textbf{\gls{lstart} PC1}: only start PC1 in this network\\
  \textbf{\gls{lhalt} [PC1]}: stop gracefully the whole network, machine by machine\\
  \textbf{\gls{lcrash} [PC1]}: stop brutally or power off the whole network\\
  \textbf{\gls{visit}}: list all the machines running\\
  \textbf{\gls{lclean}}: delete all the temporary files\\
\end{flushleft}
\end{tcolorbox}

\textbf{Attention}: \textit{lclean} is always appreciated after a \textit{lcrash} or \textit{lclean}.

\addcontentsline{toc}{subsection}{Exercise on virtual machine}
\subsection*{Exercise on virtual machine}
We are now going to do an exercise. Exercise files include "\textit{lab.conf}"(description of network), "\textit{PC1.startup}"(initialization of PC1) and "\textit{PC2.startup}".

\lstinputlisting[caption=Virtual Machine lab.conf]{programs/lab.conf}
\lstinputlisting[caption=Virtual Machine PC1.startup]{programs/PC1.startup}
\lstinputlisting[caption=Virtual Machine PC2.startup]{programs/PC2.startup}

We can do many things on this network of two virtual machines. To start the network, first enter privileged mode by using \textit{\gls{sudo -s}}, then type \textit{lstart} under the project directory.

\begin{enumerate}
  \item Test connectivity from PC1 to PC2
  \begin{itemize}
    \item type \textit{\gls{ping}} 10.0.0.2(IP address of PC2)
    \item interrupt the connection by \textit{Ctrl-C}
  \end{itemize}
  
  \item Connect to virtual PCs from the Ubuntu environment
  \begin{itemize}
    \item in the PC1 window, type \textit{\gls{passwd}} to set a password for the root account
    \item in a new terminal, type \textit{\gls{sudo ssh}} 172.16.0.1 to connect to PC1 (the password defined necessary)
  \end{itemize}
  
  \item Capture network traffic to or from a machine
  \begin{itemize}
    \item keep "ping" running
    \item in the terminal with "ssh" session, type \textit{tcpdump -i eth0 -w /hostlab/TP1-PC1-eth0.pcap}
    \item interrupt it by \textit{Ctrl-C}
  \end{itemize}
  
  \item Share files between virtual PCs and the Ubuntu environment
  \begin{itemize}
    \item run "Wireshark" and open TP1-PC1-eth0.pcap
    \item at the end of test, stop "ssh" session, stop gracefully the whole network and clear all the temporary files.
  \end{itemize}  
\end{enumerate}
\end{document}